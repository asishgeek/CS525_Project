\documentclass{article}      % Specifies the document class

\usepackage{natbib}
\usepackage{fullpage}
\usepackage{caption}
\usepackage{subcaption}
\usepackage{amsmath}
\usepackage{amsthm}
\usepackage{amsfonts}
\usepackage{float}
\usepackage{graphicx}
\DeclareGraphicsExtensions{.pdf,.png,.jpg}
\graphicspath{{figures/}}
\newcommand{\mrna}{\text{mRNA}~}
\newcommand{\micrna}{\text{microRNA}~}
\newcommand{\delg}{\ensuremath{\Delta G}~} % Delta G
\newcommand{\deldelg}{\ensuremath{\Delta \Delta G}~} %Delta Delta G
\newcommand{\mcal}[1]{\ensuremath{\mathcal{#1}}}
\newcommand{\mbf}[1]{\ensuremath{\mathbf{#1}}}
\newcommand{\mbb}[1]{\ensuremath{\mathbb{#1}}}
\DeclareMathOperator*{\argmin}{arg\,min}
\newtheorem{definition}{Definition}
\newcommand{\iref}[1]{Fig.~\ref{#1}}
\newcommand{\X}{\ensuremath{\mathbf{X}}}

\title{In-network reduction in Hadoop}  
\author{Asish Ghoshal, Agrima Jindal, Riya Charaya}      
\date{}

\bibliographystyle{apalike}

\begin{document}             % End of preamble and beginning of text.

\maketitle                   % Produces the title.

\section*{}
Hadoop is one of the most popular implementations of MapReduce and has been extensively used in solving big data problems.
The problem with processing massive amounts of data in Hadoop is that, a lot of network traffic is generated in the shuffle phase.
Each reducer is responsible for grabbing data from all the mappers, after all the map tasks have been completed, and
reducing the data locally. If there are $N$ nodes out of which $K$ participate in the reduce phase, 
then the shuffle phase has an all-to-all traffic pattern with $\mathcal{O}(NK)$ flows.
In this project we introduce in-network reduction for Hadoop wherein data is progressively reduced as it
flows within the network. Towards this end, we propose to reduce data in a tree fashion across reducers i.e. in the first phase
pairs of nodes reduce data among themselves, in the next iteration the data is further reduced and proceeds until there are $K$
messages after which data is reduced locally. Our approach is similar to that of \citep{costa2012camdoop}; while \citep{costa2012camdoop}
used a direct-connect network topology (CamCube), we plan to evaluate our approach on a variety of network topologies and compute
various performance metrics like speedup, communication cost, clock time among others.


\bibliography{proposal}

\end{document}
